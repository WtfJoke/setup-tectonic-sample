\documentclass{article}
\usepackage[utf8]{inputenc}
\usepackage[T1]{fontenc}
\usepackage[ngerman]{babel}
\usepackage{xcolor}
\usepackage{tikz}
\usepackage[paperwidth=70mm, paperheight=120mm, margin=2mm]{geometry}
\pagestyle{empty}

\begin{document}

% --- Karte 1: Zauberer-Karte ---
\begin{tikzpicture}[remember picture, overlay]
    % Rahmen mit abgerundeten Ecken
    \draw[rounded corners=5mm, line width=1.5pt, black] (0,0) rectangle (70mm,120mm);
    
    % Hintergrund mit Farbverlauf (magischer Effekt)
    \shade[rounded corners=5mm, bottom color=blue!20, top color=purple!50] (0,0) rectangle (70mm,120mm);
    
    % Titel der Karte
    \node at (35mm,110mm) {\fontsize{16}{18}\selectfont\textbf{Zauberer-Karte}};
    
    % Zentrales Designelement (großer Stern)
    \node at (35mm,70mm) {\fontsize{40}{42}\selectfont$\bigstar$};
    
    % Verstecktes Easter Egg (fadet fast in den Hintergrund)
    \node[rotate=45, opacity=0.1, scale=0.5] at (60mm,20mm) {Mario};
\end{tikzpicture}

\newpage

% --- Karte 2: Narr-Karte ---
\begin{tikzpicture}[remember picture, overlay]
    % Rahmen
    \draw[rounded corners=5mm, line width=1.5pt, black] (0,0) rectangle (70mm,120mm);
    
    % Verspielter Farbverlauf
    \shade[rounded corners=5mm, bottom color=red!20, top color=yellow!50] (0,0) rectangle (70mm,120mm);
    
    % Titel der Karte
    \node at (35mm,110mm) {\fontsize{16}{18}\selectfont\textbf{Narr-Karte}};
    
    % Dekoratives Element: Verspielter, geschwungener Strich
    \draw[line width=1pt, black] (10mm,70mm) .. controls (35mm,90mm) and (35mm,50mm) .. (60mm,70mm);
    
    % Verstecktes Easter Egg
    \node[rotate=-30, opacity=0.1, scale=0.5] at (10mm,30mm) {Julia};
\end{tikzpicture}

\newpage

% --- Karte 3: Farben-Karte ---
\begin{tikzpicture}[remember picture, overlay]
    % Rahmen
    \draw[rounded corners=5mm, line width=1.5pt, black] (0,0) rectangle (70mm,120mm);
    
    % Hintergrund in drei farbigen Bändern
    \fill[red!30] (0,0) rectangle (70mm,40mm);
    \fill[green!30] (0,40mm) rectangle (70mm,80mm);
    \fill[blue!30] (0,80mm) rectangle (70mm,120mm);
    
    % Titel der Karte
    \node at (35mm,110mm) {\fontsize{16}{18}\selectfont\textbf{Farben-Karte}};
    
    % Dekorative Kreise als zusätzliche grafische Elemente
    \draw[fill=white, opacity=0.7] (20mm,60mm) circle (5mm);
    \draw[fill=white, opacity=0.7] (50mm,60mm) circle (5mm);
    
    % Verstecktes Easter Egg
    \node[rotate=15, opacity=0.1, scale=0.5] at (50mm,10mm) {Alex};
\end{tikzpicture}

\end{document}
