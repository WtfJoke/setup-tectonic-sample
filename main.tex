\documentclass{article}
\usepackage[utf8]{inputenc}
\usepackage[T1]{fontenc}
\usepackage[ngerman]{babel}
\usepackage{lmodern}
\usepackage{geometry}
\geometry{
  paperwidth=70mm,
  paperheight=120mm,
  left=3mm,
  right=3mm,
  top=3mm,
  bottom=3mm
}
\usepackage{tikz}
\usepackage{fancyhdr}
\pagestyle{fancy}
\fancyhead{}
\fancyfoot{}
\fancyfoot[C]{\thepage}
\renewcommand{\headrulewidth}{0pt}
\renewcommand{\footrulewidth}{0pt}

\begin{document}
\thispagestyle{empty}
\begin{center}
    \Large \textbf{Wizard Fan Cards}\\[1em]
    \normalsize Erste Entwürfe der Fan-Karten für das Spiel \textit{Wizard}
\end{center}

\vspace{0.5em}

\textbf{Format:} 70x120~mm mit abgerundeten Ecken (entsprechend der Vorlage)

\vspace{0.5em}

\textbf{Besonderheiten:}
\begin{itemize}
    \item Originalgetreues Design mit kleinen kreativen Anpassungen
    \item Versteckte Easter Eggs, u. a. die Namen \textbf{Ioannis, Mario, Niels, Alex, Leo, Manuel, Julia} (jeder Name nur einmal)
    \item Leichte stilistische Anpassungen, um das Set einzigartig zu machen
\end{itemize}

\vspace{0.5em}

\textbf{Erste Karten zum Review:}
\begin{enumerate}
    \item \textbf{Zauberer-Karte:}
    \begin{itemize}
        \item Hintergrund mit leichtem, magischem Leuchteffekt
        \item Versteckter Name in den Mustern integriert
    \end{itemize}
    \item \textbf{Narr-Karte:}
    \begin{itemize}
        \item Verspieltes, leicht verändertes Design mit subtiler Textur
        \item Ein Easter Egg im unteren Randbereich
    \end{itemize}
    \item \textbf{Farben-Karten (erste zwei Prototypen):}
    \begin{itemize}
        \item Farbige Anpassungen, um sie etwas individueller wirken zu lassen
        \item Kleine Symbole als geheime Hinweise
    \end{itemize}
\end{enumerate}

\vspace{0.5em}

Bitte gib mir Feedback zu diesen Entwürfen, damit ich den restlichen Kartensatz entsprechend anpassen kann. Falls du Änderungen wünschst (z. B. bei der Farbwahl oder Anordnung der Easter Eggs), lass es mich wissen!

\end{document}
